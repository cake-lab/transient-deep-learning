\section{Workload}
\label{workload}


\begin{figure}[t]

    \begin{subfigure}{0.5\textwidth}
    \centering
    \includegraphics[width=\textwidth]{{TaskSchedulingByUser}.pdf}
        \caption{}
    \label{workload:usercompare:task_scheduled} 
    \end{subfigure}
        
    \hfill

    \begin{subfigure}{0.5\textwidth}
    \centering
    \includegraphics[width=\textwidth]{{TaskRuntimeByUser}.pdf}
        \caption{}
    \label{workload:usercompare:task_runtime} 
    \end{subfigure}
    
    
    \caption{}
    \label{eval:fig:two-models}
\end{figure}

As discussed in the Section~\ref{background.workdloads} there exist many workloads available for study but these are all at the cluster-wide scale.
This means that these workloads are useful for the purpose of designed new clusters and schedulers but not useful for provisioning as few users are able to scale up a virtual cluster to 3k cores, let alone 300k cores.
Therefore, an important first step was to isolate more representative workloads from these larger workloads.


\subsection{Workload Isolation}
\label{workload.isolation}

For a source workload we used the Google Borg workload~\cite{google_trace}.
This workload has two main advantages.
First, the workload is broken up into tasks with their user IDs anonymized in a regular manner.
This allows easy isolation of tasks belonging to each user.
Second, this workload is from the Google Compute Engine cloud so the workloads we see are from actual cloud users.
This indicates that they are what users would be be expected to run in the cloud.
\footnote{Some of these jobs are non-production, so not from cloud customers.  However, these workloads are still useful since they can allow us insight to how an outside user may transition to a cloud-based virtual cluster.}
\par

Parsing the Google Borg trace required separating tasks based on users.
The five users with the most tasks run throughout the course of the workload were identified and used for further analysis.

\subsection{Workload Analysis}
\label{workload.analysis}

Figure~\ref{eval:fig:two-models} shows the details of tasks submitted by the five most active users in the Google Trace.
\par

Figure~\ref{workload:usercompare:task_scheduled} shows the number of tasks submitted each day of the google trace.
What is particularlly important to note about this is that there is a large variation in the habits of different users.
User \texttt{29678} submits a nearly constant high number of tasks throughout the entire month.  
Meanwhile, user \texttt{b4c85} submitted a large number of tasks but not as regularly as \texttt{29678}.  Instead they submitted in a few main batches, with almost no tasks submitted in the time between.
Finally, user \texttt{4a383} submitted very few tasks throughout most of the trace but in the last week submitted over 200ktasks.
\par

Figure~\ref{workload:usercompare:task_runtime} shows the runtime distribution of the tasks submitted by each user on a logarithmic scale.
This shows that different users are running different kinds of tasks.
Users \texttt{b4c85}, \texttt{4a383} and \texttt{e6332} have profiles that seem to be nearly bimodal in distribution.
Meanwhile users \texttt{29678} and \texttt{6b308} have more complex profiles.
\par

The most interesting of these runtime distributions is that of user \texttt{b4c85} due to how well the distribution seems to be bimodal.
This trace looks like it could possibly be for a typical set of map-reduce jobs with lots of short map jobs followed by a few long reduce jobs.
Paired with its submission distribution it forms a good baseline for a user that is taking advantage of a cloud-based virtual cluster.



